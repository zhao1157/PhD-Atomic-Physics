\begin{abstract}
	Rosseland mean opacity of the solar elemental composition is a required input microphysics in the standard solar model. The opacity table used two decades ago contributed to the excellent agreement between the standard solar model and the helioseismology measurement. However, more realistic stellar atmosphere modeling suggested a drastic reduction in the solar metallicity, which caused significant discrepancies in many aspects of the Sun between the standard solar model and the helioseismology. To restore the agreement, some authors argue that the opacities should be larger. There is some experimental evidence for this. Iron opacity measured at the conditions similar to those near the solar convection zone base using the Sandia National Laboratories Z facility revealed significant underestimation of the opacity by all available theoretical models. 
	
	With the goal of obtaining more accurate Rosseland Mean Opacity we carry out the ab-initio atomic calculations for radiative data using the state-of-the-art codes including a close-coupling Breit-Pauli $R$-Matrix code and a relativistic distorted wave code. With the supercomputing power we have today, we are able to do the largest-scale Breit-Pauli $R$-Matrix calculation for \ion{Fe}{xvii} and \ion{Fe}{xviii}. To ensure the completeness and convergence of the data, we use the relativistic distorted wave calculation to ``top up" the $R$-Matrix data, and a detailed description of bridging the $R$-Matrix calculation and relativistic distorted wave calculation is presented in this dissertation.

\end{abstract}