\begin{acknowledgments}
	First of all, I would like to thank the not-too-bad China-U.S. relations over the past few years. I believe if there is a problem, there is always a GOOD way out of it. No matter who will lead the world, I hope both countries and all others can get along well with each other, till the end of the time.
	
	I would like to thank the Physics Department of the Ohio State University (OSU) for offering me an excellent opportunity to work on my Ph.D. degree in this great department, full of wonderful people and brilliant minds. Heartful thanks go to Jon Pelz and Kris Dunlap for their support and help especially during the time after I quit from Dr. Yang's condensed matter experimental group and before I joined the computational group that I am in now. I also would like to thank all the people who make this department great.
	
	My sincere thanks are due to Anil Pradhan for accepting me who had almost no computational experience as a group member, having me work on projects that helped me gain lots of programming experience, and providing one academic travel and one summer GRA financial support. I also would like to thank him and Sultana Nahar for their teaching on the atomic physics and the relevant code. I am also very grateful for Chris Orban, who gave me a quick start at the beginning of my computational research, helped me a great deal in programming, and always tried to be helpful throughout my Ph.D career. Among my Ph.D. committee, I would like to thank Amy Hill and Enam Chowdhury for their care and attention paied in my progress towards the completion of my degree. I'm also extremely grateful for Werner Eissner for his well-maintained $R$-Matrix code and for his help in fixing one fatal bug. Without him, I do not think I would make this far, and the whole project might fail. I also quite enjoyed working with him, though remotely.
	
	I would like to thank David Heisterberg at the Ohio Supercomputer Center (OSC), who helped me a lot in providing lots of information when my PBS jobs failed, especially at the beginning of $R$-Matrix calculation leading to the discovery that ``scratch'' files in Fortran consume memory at OSC, thus the a-few-hundred-GB-memory problem is resolved. I also would like to thank David Will in the Astronomy Department of OSU, especially for his kind and generosity in allocating more computing resources to my PBS jobs for a few months just before Arjuna cluster was retired. I also would like to thank Keith Stewart and Sandy Shew for their helpful guidance in leveraging ASC Unity cluster and for doubling the active PBS jobs I can have.
	
	Special thanks go to the following friends and colleagues who helped me run the large-scale $R$-Matrix calculation using their Unity computing resources (names in alphabetic order): Jiaxin Wu, Keng Yuan Meng, Max Westphal, Xiankun Li, Yonas Getachew and Zhefu Yu, without whose help I could never finish the work as planned.
	
	At last, I would love to thank my family for their constant support in the past six years and thank Shaoyun Dong for appearing in my life, accompanying me for the past more than a year, and helping me concentrate on the important things in my life. I hope you and I can spend the rest of our lives together, till the end of the time.
	
\end{acknowledgments}